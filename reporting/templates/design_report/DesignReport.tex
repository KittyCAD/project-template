% --------------------------------------------------------------------------------------
% DOCUMENT METADATA - TO BE FILLED BY USER
% --------------------------------------------------------------------------------------
% --- Project Identification ---
\newcommand{\ProjectRef}{YYP0N}                         % e.g., 25P01
\newcommand{\ProjectTitle}{PROJECT TITLE}               % e.g., MORSE KEY

% --- Report Identification ---
\newcommand{\ReportRef}{R0N}                            % e.g., R02
\newcommand{\ReportTitle}{INITIAL DESIGN DOCUMENTATION} % e.g., INITIAL DESIGN DOCUMENTATION
\newcommand{\DocVersion}{V0N}                           % e.g., V01

% --- Authoring and Assets ---
\newcommand{\AuthorName}{AUTHOR NAME}                   % e.g., Your Name
\newcommand{\ReleaseDate}{\today}                       % Use \today or specify a date e.g., January 1, 2024
\newcommand{\LogoPath}{./assets/logo.pdf}               % Path to your logo file (ensure ./assets/ exists or change path)
\newcommand{\ReferenceFile}{DesignReport.bib}           % Name of your BibTeX file (ensure this file exists)
% --------------------------------------------------------------------------------------
% END DOCUMENT METADATA
% --------------------------------------------------------------------------------------


% --------------------------------------------------------------------------------------
% DOCUMENT SETUP — GENERALLY TO BE LEFT UNCHANGED
% --------------------------------------------------------------------------------------
% Assemble shorthand
\newcommand{\ProjectFullRef}{\ProjectRef{} \ProjectTitle{}}
\newcommand{\ReportFullRef}{\ReportRef{}-\DocVersion{}}
\newcommand{\ReportFullTitle}{\ProjectRef{}-\ReportRef{}\DocVersion{}: \ReportTitle{}}


% Fonts
\documentclass[10pt]{article}

\usepackage[utf8]{inputenc}
\usepackage[OT1]{fontenc}
\usepackage{ocr}

\usepackage{lmodern}
\renewcommand*\familydefault{\sfdefault}

% Set sections to use OCR-A1
\usepackage{sectsty}
\sectionfont{\ocrfamily\Large}
\subsectionfont{\ocrfamily\large}
\subsubsectionfont{\ocrfamily\normalsize}

% Core packages
\usepackage[english]{datetime2}
\usepackage{amsmath}
\usepackage{amssymb}
\usepackage{array}
\usepackage{booktabs}
\usepackage{enumitem}
\usepackage{fancyhdr}
\usepackage{float}
\usepackage{geometry}
\usepackage{graphicx}
\usepackage{longtable}
\usepackage{textcomp}
\usepackage{titlesec}
\usepackage{titling}
\usepackage{xcolor}

% Links
\usepackage[colorlinks=true,
            linkcolor=gray,
            citecolor=gray,
            urlcolor=gray
           ]{hyperref}

% Date
\DTMsetdatestyle{iso}

% Bibliography
\usepackage[backend=bibtex, style=ieee, citestyle=numeric-comp]{biblatex}
\addbibresource{\ReferenceFile}


% Set narrow page margins
\geometry{
  paper=a4paper,
  top=0.75in,
  bottom=0.75in,
  left=0.55in,
  right=0.55in,
  headsep=0.25in
}

% Remove paragraph indentation, add spacing between paragraphs
\usepackage{parskip}
\setlength{\parindent}{0pt}
\setlength{\parskip}{6pt}

% Header/Footer setup
\pagestyle{fancy}
\fancyhf{}
\renewcommand{\headrulewidth}{0.4pt}
\renewcommand{\footrulewidth}{0pt}
\fancyfoot[C]{\thepage}
\fancyhead[L]{\ocrfamily\small\ProjectFullRef}
\fancyhead[R]{\ocrfamily\small\ReportFullRef}
\setlength{\headheight}{20pt}

% Custom section style with horizontal lines
\titleformat{\section}
  {\ocrfamily\Large\bfseries}
  {\thesection}{1em}{}
  \titlespacing*{\section}
  {0pt}{1.5em}{1em}

\titleformat{\subsection}
  {\ocrfamily\large\bfseries}
  {\thesubsection}{1em}{}

\titleformat{\subsubsection}
  {\ocrfamily\normalsize\bfseries}
  {\thesubsubsection}{1em}{}

% Create a custom title command
\newcommand{\customtitle}{%
  \noindent
  \begin{minipage}[t]{0.65\textwidth}
    \vspace{-0.5cm}
    {\ocrfamily\Large\bfseries DESIGN REPORT \par} % Changed Title
  \end{minipage}%
  \begin{minipage}[t]{0.35\textwidth}
    \flushright{}
    \includegraphics[width=0.5\textwidth]{\LogoPath}
  \end{minipage}

  \vspace{0.3cm}
  \hrule height 0.8pt
  \vspace{0.3cm}

  {\ocrfamily\bfseries\ProjectFullRef\par}
  {\ocrfamily\large\bfseries\ReportTitle\par}

  \vspace{0.5em}

  \begin{tabular}{@{}ll@{\hspace{2cm}}ll@{}}
    \ocrfamily\textbf{REPORT REF:} & \ocrfamily \ReportRef &
    \ocrfamily\textbf{AUTHOR:}     & \ocrfamily \AuthorName \\

    \ocrfamily\textbf{VERSION:}    & \ocrfamily \DocVersion &
    \ocrfamily\textbf{DATE:}       & \ocrfamily \ReleaseDate \\
  \end{tabular}

  \vspace{0.3cm}
  \hrule height 0.8pt
  \vspace{0.25cm}
}


% Create fullwidth table command
\newenvironment{fullwidthtable}
  {\begin{center}
   \begin{tabular*}{\textwidth}{@{\extracolsep{\fill}}ll@{}}}
  {\end{tabular*}
   \end{center}}


% Fold URLs
\usepackage{xurl}
\setlength{\emergencystretch}{1em}

% --------------------------------------------------------------------------------------
% END DOCUMENT SETUP
% --------------------------------------------------------------------------------------


\begin{document}
\vspace*{-1cm}
\thispagestyle{plain}
\customtitle{}

% --------------------------------------------------------------------------------------
% REPORT CONTENT
% --------------------------------------------------------------------------------------

\section{Summary}
% TODO: Add summary content here.
% Briefly describe the purpose, key findings, and conclusions of the report.

Example reference \cite{hibbeler2016mechanics}...

\section{Brief}
% TODO: Add content introducing the project brief.
% Outlines the initial understanding of the project goals and context.

\subsection{Product Description}
% TODO: Add content describing the product.
% A concise overview of the item or system being designed.

\subsection{Initial Requirements}
% TODO: Add content detailing the initial requirements.
% List the primary functional and non-functional needs identified at the outset.
The product must:
\begin{itemize}[leftmargin=*] % Example bulleted list
	\item % TODO: Add requirement
\end{itemize}

\subsection{Constraints}
% TODO: Add content detailing the project constraints.
% List limitations or boundaries affecting the design (e.g., cost, materials, regulations).
The product must:
\begin{itemize}[leftmargin=*] % Example bulleted list
	\item % TODO: Add constraint
\end{itemize}

\section{Research \& Requirements Development}
% TODO: Add content introducing the research phase.
% Describes the background research undertaken and the resulting detailed requirements.

\subsection{Key Topics}
% TODO: Add content listing key research areas.
% Identify the main subjects investigated (e.g., existing solutions, user needs, technology).

\subsection{Consolidated Requirements}
% TODO: Add content introducing the finalized requirements.
% Presents the refined and categorized requirements derived from the brief and research.

\subsubsection{Performance Requirements}
% TODO: Define specific, measurable performance targets.
P1. \\
P2.

\subsubsection{Electrical Requirements}
% TODO: Define electrical specifications, interfaces, or standards.
E1. \\
E2.

\subsubsection{Mechanical Requirements}
% TODO: Define mechanical properties, loads, or geometric needs.
M1. \\ % TODO: Add requirement text
M2. % TODO: Add requirement text

\subsubsection{Manufacturing Constraints}
% TODO: Detail constraints related to production processes or capabilities.
F1. \\ % TODO: Add requirement text
F2. % TODO: Add requirement text

\subsubsection{Integration}
% TODO: Define requirements for interaction with other systems or components.
I1. \\ % TODO: Add requirement text
I2. % TODO: Add requirement text

\section{Concept Development} \label{sec:concept-development}
% TODO: Add content introducing the concept development phase.
% Describes the exploration and selection of potential design solutions.

\subsection{Concept Exploration}
% TODO: Add content describing the concepts considered.
% Detail the different design ideas or approaches investigated.

\subsubsection{Concept 1}
% TODO: Describe the first design concept.

\subsubsection{Concept 2}
% TODO: Describe the second design concept.

% Example commented-out figure inclusion:
% \begin{figure}[H] % [H] forces placement here (requires float package)
%   \centering
%   \includegraphics[width=0.75\textwidth]{./assets/concept_figure.png} % Adjust path and width
%   \caption{Caption for concept figure.}
%   \label{fig:concept-figure-label} % Label for cross-referencing (\autoref{fig:concept-figure-label})
% \end{figure}

\subsection{Concept Selection} \label{sec:concept-selection}
% TODO: Explain the rationale for choosing the final concept.
% Describe the evaluation process and justification for the selected design direction.

\section{Design Development}
% TODO: Add content introducing the detailed design phase.
% Describes the process of refining the selected concept into a detailed specification.

\subsection{Key Design Features}
% TODO: Describe the important elements and characteristics of the final design.
% Detail specific components, geometries, or functionalities.

\subsection{Material Selection}
% TODO: Justify the choice of materials for the design.
% Explain why specific materials were selected based on requirements and properties.

\subsection{Manufacturing \& Assembly Considerations}
% TODO: Discuss how the design will be produced and put together.
% Detail manufacturing processes, assembly steps, tolerances, and potential challenges.

\subsection{Cost Considerations}
% TODO: Analyze the cost implications of the design.
% Discuss material costs, manufacturing expenses, and overall economic feasibility.

\section{Design Validation}
% TODO: Add content introducing the validation phase.
% Describes the methods used to verify that the design meets requirements.

\subsection{Analytical Methods}
% TODO: Describe calculations or theoretical analysis performed.
% Detail any hand calculations, spreadsheets, or analytical models used for validation.

\subsection{FEA/CFD} % Or other simulation methods
% TODO: Describe the simulation approach used.
% Introduce the use of computational methods like Finite Element Analysis or Computational Fluid Dynamics.

\subsubsection{Methodology}
% TODO: Detail the setup of the simulation model.
% Describe mesh, boundary conditions, loads, material models, and assumptions.

\subsubsection{Mesh Convergence}
% TODO: Describe the study ensuring mesh independence (if performed).
% Explain how mesh density effects on results were assessed.

\subsubsection{Results}
% TODO: Present and interpret the simulation outcomes.
% Show key results (e.g., stress, deformation, flow) and compare them to requirements or analytical predictions.

\section{Conclusion}
% TODO: Add content summarizing the report's findings.
% Provides a final overview of the design process and outcomes.

\subsection{Summary}
% TODO: Briefly reiterate the main conclusions of the design process.
% Restate the final design status and key performance indicators.

\subsection{Limitations} \label{sec:limitations}
% TODO: Identify the limitations of the current design or analysis.
% Acknowledge assumptions, simplifications, or areas not fully addressed.

\subsubsection{Material Behaviour}
% TODO: List limitations related to material modeling or properties.

\subsubsection{Analysis}
% TODO: List limitations related to the analytical or simulation methods used.

\subsubsection{Manufacturing}
% TODO: List limitations related to manufacturing assumptions or processes.

\subsection{Further Work}
% TODO: Suggest potential next steps or areas for future development.
% Recommend improvements, further testing, or refinements based on limitations or future goals.

\section{Appendices}
% TODO: Add supplementary information not essential to the main body.
% Contains supporting data, detailed calculations, meeting minutes, etc.

\subsection{Appendix A: Design Reviews} % TODO: Adjust Appendix title/content as needed
% TODO: Detail the outcomes of design review meetings.

\subsubsection{Design Review \#1}
Conducted on: YYYY-MM-DD

Key outcomes:
\begin{itemize}[leftmargin=*]
	\item % TODO: Add outcome
	      %\item % TODO: Add outcome
\end{itemize}

\section{References}
% TODO: List all cited works.
% Bibliography is automatically generated here by \printbibliography.
\printbibliography[heading=none] % 'heading=none' because we have \section{References}

% --------------------------------------------------------------------------------------
% END DOCUMENT BODY
% --------------------------------------------------------------------------------------
\end{document}