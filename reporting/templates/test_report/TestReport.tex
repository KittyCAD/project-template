% --------------------------------------------------------------------------------------
% DOCUMENT METADATA - TO BE FILLED BY USER
% --------------------------------------------------------------------------------------
% --- Project Identification ---
\newcommand{\ProjectRef}{YYP0N}               % e.g., 25P01
\newcommand{\ProjectTitle}{PROJECT TITLE}     % e.g., MORSE KEY

% --- Report Identification ---
\newcommand{\ReportRef}{R0N}                  % e.g., R02
\newcommand{\ReportTitle}{TEST REPORT TITLE}  % e.g., EXPERIMENTAL DESIGN VALIDATION
\newcommand{\DocVersion}{V0N}                 % e.g., V01

% --- Authoring and Assets ---
\newcommand{\AuthorName}{AUTHOR NAME}         % e.g., Your Name
\newcommand{\ReleaseDate}{\today}             % Use \today or specify a date e.g., January 1, 2024
\newcommand{\LogoPath}{./assets/logo.pdf}     % Path to your logo file (ensure ./assets/ exists or change path)
\newcommand{\ReferenceFile}{TestReport.bib}   % Name of your BibTeX file (ensure this file exists)
% --------------------------------------------------------------------------------------
% END DOCUMENT METADATA
% --------------------------------------------------------------------------------------


% --------------------------------------------------------------------------------------
% DOCUMENT SETUP — GENERALLY TO BE LEFT UNCHANGED
% --------------------------------------------------------------------------------------
% Assemble shorthand
\newcommand{\ProjectFullRef}{\ProjectRef{} \ProjectTitle{}}
\newcommand{\ReportFullRef}{\ReportRef{}-\DocVersion{}}
\newcommand{\ReportFullTitle}{\ProjectRef{}-\ReportRef{}\DocVersion{}: \ReportTitle{}}


% Fonts
\documentclass[10pt]{article}

\usepackage[utf8]{inputenc}
\usepackage[OT1]{fontenc}
\usepackage{ocr}      

\usepackage{lmodern}
\renewcommand*\familydefault{\sfdefault}

% Set sections to use OCR-A1
\usepackage{sectsty}
\sectionfont{\ocrfamily\Large}
\subsectionfont{\ocrfamily\large}
\subsubsectionfont{\ocrfamily\normalsize}

% Core packages
\usepackage[english]{datetime2}
\usepackage{amsmath}
\usepackage{amssymb}
\usepackage{array}
\usepackage{booktabs}
\usepackage{enumitem}
\usepackage{fancyhdr}
\usepackage{float}
\usepackage{geometry}
\usepackage{graphicx}
\usepackage{longtable}
\usepackage{textcomp}
\usepackage{titlesec}
\usepackage{titling}
\usepackage{xcolor}

% Links
\usepackage[colorlinks=true,
            linkcolor=gray,
            citecolor=gray,
            urlcolor=gray
           ]{hyperref}

% Date
\DTMsetdatestyle{iso}

% Bibliography
\usepackage[backend=bibtex, style=ieee, citestyle=numeric-comp]{biblatex}
\addbibresource{\ReferenceFile}


% Set narrow page margins
\geometry{
  paper=a4paper,
  top=0.75in,
  bottom=0.75in,
  left=0.55in,
  right=0.55in,
  headsep=0.25in
}

% Remove paragraph indentation, add spacing between paragraphs
\usepackage{parskip}
\setlength{\parindent}{0pt}
\setlength{\parskip}{6pt}

% Header/Footer setup
\pagestyle{fancy}
\fancyhf{}
\renewcommand{\headrulewidth}{0.4pt}
\renewcommand{\footrulewidth}{0pt}
\fancyfoot[C]{\thepage}
\fancyhead[L]{\ocrfamily\small\ProjectFullRef}
\fancyhead[R]{\ocrfamily\small\ReportFullRef}
\setlength{\headheight}{20pt}

% Custom section style with horizontal lines
\titleformat{\section}
  {\ocrfamily\Large\bfseries}
  {\thesection}{1em}{}
  \titlespacing*{\section}
  {0pt}{1.5em}{1em}

\titleformat{\subsection}
  {\ocrfamily\large\bfseries}
  {\thesubsection}{1em}{}

\titleformat{\subsubsection}
  {\ocrfamily\normalsize\bfseries}
  {\thesubsubsection}{1em}{}

% Create a custom title command
\newcommand{\customtitle}{%
  \noindent
  \begin{minipage}[t]{0.65\textwidth}
    \vspace{-0.5cm}
    {\ocrfamily\Large\bfseries TEST REPORT \par}
  \end{minipage}%
  \begin{minipage}[t]{0.35\textwidth}
    \flushright{}
    \includegraphics[width=0.5\textwidth]{\LogoPath}
  \end{minipage}

  \vspace{0.3cm}
  \hrule height 0.8pt
  \vspace{0.3cm}

  {\ocrfamily\bfseries\ProjectFullRef\par}
  {\ocrfamily\large\bfseries\ReportTitle\par}

  \vspace{0.5em}

  \begin{tabular}{@{}ll@{\hspace{2cm}}ll@{}}
    \ocrfamily\textbf{REPORT REF:}   & \ocrfamily \ReportRef &
    \ocrfamily\textbf{AUTHOR:}    & \ocrfamily \AuthorName \\

    \ocrfamily\textbf{VERSION:} & \ocrfamily \DocVersion &
    \ocrfamily\textbf{DATE:}       & \ocrfamily \ReleaseDate \\
  \end{tabular}

  \vspace{0.3cm}
  \hrule height 0.8pt
  \vspace{0.25cm}
}


% Create fullwidth table command (Note: not used by translated content, retained for consistency)
\newenvironment{fullwidthtable}
  {\begin{center}
   \begin{tabular*}{\textwidth}{@{\extracolsep{\fill}}ll@{}}}
  {\end{tabular*}
   \end{center}}


% Fold URLs
\usepackage{xurl}  % Allows URLs to break at any character
\setlength{\emergencystretch}{1em}  % Adds extra stretch to avoid overfull boxes

\begin{document}
\vspace*{-1cm}
\thispagestyle{plain}
\customtitle{}

% --------------------------------------------------------------------------------------
% END DOCUMENT SETUP
% --------------------------------------------------------------------------------------


% --------------------------------------------------------------------------------------
% REPORT CONTENT
% --------------------------------------------------------------------------------------
% Replace the comments below with your actual content.

\section{Summary}
% TODO: Add summary content here.
% Briefly describe the purpose and main findings of the report.

Example reference \cite{hibbeler2016mechanics}...

\section{Objectives}
% TODO: Add objectives content here.
% List the specific goals of the work described in this report.

\section{Equipment / Setup / Materials} % Renamed slightly for broader use
% TODO: Add equipment/setup/materials content here.
% Describe the tools, software, materials, or setup used.
% You can include figures using the \includegraphics command within a figure environment:
% \begin{figure}[H] % [H] forces placement here (requires float package)
%   \centering
%   \includegraphics[width=0.75\textwidth]{./path/to/your/image.png} % Adjust path and width
%   \caption{Caption for your figure.}
%   \label{fig:your-figure-label} % Label for cross-referencing (\autoref{fig:your-figure-label})
% \end{figure}

\section{Method / Procedure}
% TODO: Add method/procedure content here.
% Detail the steps taken to achieve the objectives. Use lists if appropriate:
% \begin{enumerate}
%   \item First step.
%   \item Second step.
% \end{enumerate}
% Or
% \begin{itemize}
%   \item Bullet point 1.
%   \item Bullet point 2.
% \end{itemize}

\section{Results}
% TODO: Add results content here.
% Present the findings of your work. Include data, tables, graphs, etc.
% Example of referencing a figure: See \autoref{fig:your-figure-label}.
% Example of referencing a citation: As shown by \cite{citation_key}.

\section{Discussion} % Optional section, add if needed
% TODO: Add discussion content here.
% Interpret the results, discuss limitations, suggest implications or future work.

\section{Conclusion} % Optional section, add if needed
% TODO: Add conclusion content here.
% Summarize the key outcomes in relation to the objectives.

% --- REFERENCES SECTION ---
\section{References}
% The bibliography will be printed here based on citations in the text
% and the content of your .bib file (\ReferenceFile).
\printbibliography[heading=none] % 'heading=none' because we have \section{References}

% --------------------------------------------------------------------------------------
% END DOCUMENT BODY
% --------------------------------------------------------------------------------------
\end{document}