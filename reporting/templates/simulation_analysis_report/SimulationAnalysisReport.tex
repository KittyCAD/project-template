% --------------------------------------------------------------------------------------
% DOCUMENT METADATA - TO BE FILLED BY USER
% --------------------------------------------------------------------------------------
% --- Project Identification ---
\newcommand{\ProjectRef}{YYP0N}                           % e.g., 25P01
\newcommand{\ProjectTitle}{PROJECT TITLE}                 % e.g., MORSE KEY

% --- Report Identification ---
\newcommand{\ReportRef}{R0N}                              % e.g., R02
\newcommand{\ReportTitle}{REPORT TITLE}                   % e.g., COMPONENT CFD STUDY
\newcommand{\DocVersion}{V0N}                             % e.g., V01

% --- Authoring and Assets ---
\newcommand{\AuthorName}{AUTHOR NAME}                     % e.g., Your Name
\newcommand{\ReleaseDate}{\today}                         % Use \today or specify a date e.g., January 1, 2024
\newcommand{\LogoPath}{./assets/logo.pdf}                 % Path to your logo file (ensure ./assets/ exists or change path)
\newcommand{\ReferenceFile}{SimulationAnalysisReport.bib} % Name of your BibTeX file (ensure this file exists)
% --------------------------------------------------------------------------------------
% END DOCUMENT METADATA
% --------------------------------------------------------------------------------------


% --------------------------------------------------------------------------------------
% DOCUMENT SETUP — GENERALLY TO BE LEFT UNCHANGED
% --------------------------------------------------------------------------------------
% Assemble shorthand
\newcommand{\ProjectFullRef}{\ProjectRef{} \ProjectTitle{}} % Kept for header/footer
\newcommand{\ReportFullRef}{\ReportRef{}-\DocVersion{}}     % Kept for header/footer


% Fonts
\documentclass[10pt]{article}

\usepackage[utf8]{inputenc}
\usepackage[OT1]{fontenc}
\usepackage{ocr}

\usepackage{lmodern}
\renewcommand*\familydefault{\sfdefault}

% Set sections to use OCR-A1
\usepackage{sectsty}
\sectionfont{\ocrfamily\Large}
\subsectionfont{\ocrfamily\large}
\subsubsectionfont{\ocrfamily\normalsize} % Note: The new structure uses \subsection, not \subsubsection

% Core packages
\usepackage[english]{datetime2}
\usepackage{amsmath}
\usepackage{amssymb}
\usepackage{array}
\usepackage{booktabs}
\usepackage{enumitem}
\usepackage{fancyhdr}
\usepackage{float}
\usepackage{geometry}
\usepackage{graphicx}
\usepackage{longtable}
\usepackage{textcomp}
\usepackage{titlesec}
\usepackage{titling}
\usepackage{xcolor}

% Links
\usepackage[colorlinks=true,
            linkcolor=gray,
            citecolor=gray,
            urlcolor=gray
           ]{hyperref}

% Date
\DTMsetdatestyle{iso}

% Bibliography
\usepackage[backend=bibtex, style=ieee, citestyle=numeric-comp]{biblatex}
\addbibresource{\ReferenceFile}


% Set narrow page margins
\geometry{
  paper=a4paper,
  top=0.75in,
  bottom=0.75in,
  left=0.55in,
  right=0.55in,
  headsep=0.25in
}

% Remove paragraph indentation, add spacing between paragraphs
\usepackage{parskip}
\setlength{\parindent}{0pt}
\setlength{\parskip}{6pt}

% Header/Footer setup
\pagestyle{fancy}
\fancyhf{}
\renewcommand{\headrulewidth}{0.4pt}
\renewcommand{\footrulewidth}{0pt}
\fancyfoot[C]{\thepage}
\fancyhead[L]{\ocrfamily\small\ProjectFullRef} % Uses ProjectRef and ProjectTitle
\fancyhead[R]{\ocrfamily\small\ReportFullRef}  % Uses ReportRef and DocVersion
\setlength{\headheight}{20pt}

% Custom section style with horizontal lines
\titleformat{\section}
  {\ocrfamily\Large\bfseries}
  {\thesection}{1em}{}
  \titlespacing*{\section}
  {0pt}{1.5em}{1em}

% Adjusted title format for subsections to match sections visually (no number for subsection)
\titleformat{\subsection}
  {\ocrfamily\large\bfseries}
  {\thesubsection}{1em}{} % Keeps numbering for subsections
  \titlespacing*{\subsection}
  {0pt}{1.25em}{0.75em} % Adjust spacing if needed

% Note: \subsubsection format kept in case it's needed later
\titleformat{\subsubsection}
  {\ocrfamily\normalsize\bfseries}
  {\thesubsubsection}{1em}{}

% Create a custom title command
\newcommand{\customtitle}{%
  \noindent
  \begin{minipage}[t]{0.65\textwidth}
    \vspace{-0.5cm}
    {\ocrfamily\Large\bfseries SIMULATION/ANALYSIS REPORT \par} 
  \end{minipage}%
  \begin{minipage}[t]{0.35\textwidth}
    \flushright{}
    \includegraphics[width=0.5\textwidth]{\LogoPath}
  \end{minipage}

  \vspace{0.3cm}
  \hrule height 0.8pt
  \vspace{0.3cm}

  {\ocrfamily\bfseries\ProjectFullRef\par}
  {\ocrfamily\large\bfseries\ReportTitle\par}

  \vspace{0.5em}

  \begin{tabular}{@{}ll@{\hspace{2cm}}ll@{}}
    \ocrfamily\textbf{REPORT REF:} & \ocrfamily \ReportRef &
    \ocrfamily\textbf{AUTHOR:}     & \ocrfamily \AuthorName \\

    \ocrfamily\textbf{VERSION:}    & \ocrfamily \DocVersion &
    \ocrfamily\textbf{DATE:}       & \ocrfamily \ReleaseDate \\
  \end{tabular}

  \vspace{0.3cm}
  \hrule height 0.8pt
  \vspace{0.25cm}
}

% Create fullwidth table command (Note: not used by translated content, retained for consistency)
\newenvironment{fullwidthtable}
  {\begin{center}
   \begin{tabular*}{\textwidth}{@{\extracolsep{\fill}}ll@{}}}
  {\end{tabular*}
   \end{center}}


% Fold URLs
\usepackage{xurl}  % Allows URLs to break at any character
\setlength{\emergencystretch}{1em}  % Adds extra stretch to avoid overfull boxes

\begin{document}
\vspace*{-1cm}
\thispagestyle{plain}
\customtitle{}

% --------------------------------------------------------------------------------------
% END DOCUMENT SETUP
% --------------------------------------------------------------------------------------

% --------------------------------------------------------------------------------------
% REPORT CONTENT - NEW STRUCTURE APPLIED
% --------------------------------------------------------------------------------------
% Replace the comments below with your actual content, following the new structure.

\section{Summary \& Recommendations}
% TODO: Provide a concise overview of the report's key findings and suggest actionable recommendations based on the analysis.

\section{Introduction \& Objectives}
% TODO: Introduce the project context, state the purpose of the simulation/analysis, and list the specific objectives.

\section{Background Research}
% TODO: Summarise relevant prior work, theoretical concepts, or existing solutions that inform this report. Cite sources appropriately \cite{key}.

\section{Methodology}
% TODO: Describe the approach, methods, models, tools, and parameters used for the simulation or analysis. Detail the steps taken.

\section{Results}
% TODO: Present the data and findings obtained from the simulation or analysis. Use figures, tables, and graphs as needed.
% Example figure:
% \begin{figure}[H]
%   \centering
%   \includegraphics[width=0.75\textwidth]{./path/to/your/image.png}
%   \caption{Caption for your results figure.}
%   \label{fig:results-example}
% \end{figure}

\section{Analysis}
% TODO: Interpret the results presented in the previous section. 
% Discuss their significance, compare them to expectations or background research, and identify trends or patterns.

\section{Conclusion}
% TODO: This section summarises the entire effort. The subsections provide specific details.

\subsection{Summary}
% TODO: Briefly reiterate the main conclusions drawn from the analysis in relation to the objectives.

\subsection{Limitations}
% TODO: Discuss any constraints, assumptions, or limitations of the methodology, data, or analysis conducted.

\subsection{Further Work}
% TODO: Suggest potential next steps, improvements, or future research directions based on the report's findings and limitations.

\section{Appendices}
% TODO: Include supplementary material here, such as detailed data tables, code snippets, or extended derivations.
% Use \appendix command before appendix sections if desired for separate numbering (e.g., Appendix A).

% --- REFERENCES SECTION ---
% Kept from original template, assuming references are still needed.
\section{References}
% The bibliography will be printed here based on citations in the text
% and the content of your .bib file (\ReferenceFile).
\printbibliography[heading=none] % 'heading=none' because we have \section{References}

% --------------------------------------------------------------------------------------
% END DOCUMENT BODY
% --------------------------------------------------------------------------------------
\end{document}